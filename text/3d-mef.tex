\subsection{Método dos elementos finitos}
{
	A partir da década de 70, o método dos elementos finitos começou a ser utilizado na resolução de problemas da dinâmica dos fluidos, porém a indústria aeronáutica já utilizava desde o final da década de 50.  
	
	O método dos elementos finitos possibilita a resolução de problemas que envolvem domínios geométricos complexos pela discretização deste domínio em elementos não estruturados. Além disto permite uma implementação geral e flexível. 
	
	Aproximações de Elementos Finitos padrão são baseados no método dos resíduos ponderados ou na formulação variacional correspondente por meio da formulação de Galerkin. Porém, como dito anteriormente, o método de Galerkin apresenta instabilidade e portanto não foi utilizado no presente trabalho.
}